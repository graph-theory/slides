%!TEX root = slides.tex

\title{Graph Theory}
\subtitle{}
\author{ian.mcloughlin@gmit.ie}
\date{}


\begin{frame}
	\titlepage
\end{frame}

\begin{frame}
	\frametitle{Topics}
	\tableofcontents
\end{frame}

\section{Ionic}

\begin{frame}{Ionic Framework}
  \begin{description}
		\item[Ionic] is a cross-platform mobile development framework built on top of Cordova and Node.js.
    \vspace{0.25cm}
		\item[Cordova] Cordova is an open-source fork of PhoneGap.
    \vspace{0.25cm}
		\item[Protocol] Node is a JavaScript interpreter for the command line.
  \end{description}
\end{frame}


\begin{frame}[fragile]{JavaScript Example}
  \begin{minted}{javascript}
var groceries = ["Milk", 'Eggs', 7 + "up"];

for (var i = 0; i < groceries.length; i++)
  console.log(groceries[i]);

if (groceries.length == 0)
  console.log("The groceries list is empty.");
else if (groceries.length == 1)
  console.log("Groceries contains 1 item.");
else
  console.log("Groceries contains " + groceries.length
                              + " items.");
  \end{minted}
\end{frame}


\begin{frame}[fragile]{HTML Example}
  \begin{minted}{html}
<!DOCTYPE html>
<html>
  <head>
    <meta charset="UTF-8">
    <title></title>
    <style type="text/css">
    </style>
  </head>
  <body>
    <p>Hello, world!</p>
    <script type="text/javascript">
    </script>
  </body>
</html>
  \end{minted}
\end{frame}


\section{JavaScript}

\section{AngularJS}

\section{Compiling Apps}

\section{Templates}

\section{States}

\section{Factories}