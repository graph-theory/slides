%!TEX root = slides.tex


\section{Graph databases}


\begin{frame}{Digraph definition}
  \begin{definition}
    A \emph{digraph} (short for directional graph) consists of a finite set $V$ and a set $E$ of ordered pairs of elements of $V$.
  \end{definition}
  \vspace{0.25cm}
  \begin{description}
    \item[Degrees] of vertices can now be split into in-degrees and out-degrees.
    \item[Walks, paths, cycles] must be redefined.
    \item[Loops] are allowed in the above definition, unless we rule them out.
  \end{description}
\end{frame}


\begin{frame}{Digraph example}
  \begin{center}
    \begin{tikzpicture}
    \begin{scope}[every node/.style={circle,thick,draw}]
    \node (a) at (0,0) {$a$};
    \node (b) at (0,3) {$b$};
    \node (c) at (3,3) {$c$};
    \node (d) at (3,0) {$d$};
    \end{scope}
    \begin{scope}[every edge/.style={draw=black,thick}]
    \path (b) edge[->,> = latex'] (c);
    \path (b) edge[->,> = latex'] (d);
    \path (d) edge[->,> = latex'] (c);
    \path (c) edge[->,> = latex'] (a);
    \path (d) edge[->,> = latex'] (a);
    \end{scope}
    \end{tikzpicture}
  \end{center}
\end{frame}

\begin{frame}{Neo4j}
    \begin{columns}
      \begin{column}{0.25\textwidth}
        \includegraphics[height=1.8in]{img/neo4j.png}
      \end{column}
      \begin{column}{0.6\textwidth}
        \begin{itemize}
          \item Neo4j is an open-source NoSQL graph database implemented in Java and Scala.
          \vspace{0.25cm}
          \item Development started in 2003, it has been publicly available since 2007
          \vspace{0.25cm}
          \item Available on GitHub.
          \vspace{0.25cm}
          \item A graph is composed of two elements: a node and a relationship.
        \end{itemize}
      \end{column}
    \end{columns}
    \citeurl{http://neo4j.com/developer/graph-database/}
\end{frame}


\begin{frame}{Cypher}
  \begin{columns}
    \begin{column}{0.25\textwidth}
      \includegraphics[height=1.8in]{img/neo4j.png}
    \end{column}
    \begin{column}{0.6\textwidth}
      \begin{itemize}
        \item Cypher is a declarative graph query language.
        \vspace{0.25cm}
        \item What to retrieve from a graph, not on how to retrieve it.
        \vspace{0.25cm}
        \item Allows for expressive and efficient querying and updating of the graph store.
        \vspace{0.25cm}
        \item Cypher borrows its structure from SQL.
      \end{itemize}
    \end{column}
  \end{columns}
  \citeurl{http://neo4j.com/docs/stable/cypher-introduction.html}
\end{frame}


\begin{frame}[fragile]{Cypher: Nodes}
  Create a node with the label User, and two properties:
  \begin{minted}[linenos, frame=lines, framesep=2mm]{cypher}
CREATE (user:User { Id: 123, Name: "Jim" });
  \end{minted}
  \vspace{1cm}
  Find the node(s) with label User and their Id property being 123:
  \begin{minted}[linenos, frame=lines, framesep=2mm]{cypher}
MATCH (user:User)
WHERE user.Id = 123
RETURN user;
  \end{minted}
  \citeurl{neo4j.com/docs/stable/cypher-query-lang.html}
\end{frame}

\begin{frame}[fragile]{Cypher: Relationships}
  Create a relationship with label FOLLOWS from user(s) with Id 123 to user(s) with Id 456: 
  \begin{minted}[linenos, frame=lines, framesep=2mm]{cypher}
MATCH (user1:User), (user2:User)
WHERE user1.Id = 123 AND user2.Id = 456
CREATE user1-[:FOLLOWS]->user2;
  \end{minted}
  \citeurl{neo4j.com/docs/stable/cypher-query-lang.html}
\end{frame}

\begin{frame}[fragile]{Cypher: Relationships and Nodes}
  Create a relationship with label INVITED from user(s) with Id 123 to a new user with Id 789 and Name Jack: 
  \begin{minted}[linenos, frame=lines]{cypher}
MATCH (invitee:User)
WHERE invitee.Id = 123
CREATE invitee-[:INVITED]->(invited:User {Id: 789,
                                        Name: "Jack"});
  \end{minted}
  \citeurl{neo4j.com/docs/stable/cypher-query-lang.html}
\end{frame}


\begin{frame}[fragile]{Cypher: DELETE}
  Delete all nodes: 
  \begin{minted}[linenos, frame=lines]{cypher}
MATCH (x)
DELETE x;
  \end{minted}
  \citeurl{neo4j.com/docs/stable/cypher-query-lang.html}
\end{frame}

\begin{frame}{Labels and properties}
  \begin{itemize}
    \item Suppose we have nodes representing people.
    \item We give them the label People.
    \item We also want to identify each person as either Male or Female.
    \item Should we use Male and Female labels, or a Gender property?
    \item If you are going to use the person's gender in a lot of queries, a normal property will be relatively slow, so you should use a label.
    \item However, you can also index some of your properties to high-light them as important.
    
  \end{itemize}
  \citeurl{neo4j.com/docs/stable/cypher-query-lang.html}
\end{frame}

\begin{frame}[fragile]{Cypher: shortestPath}
  Find the minimum number of hops between two nodes.
  \begin{minted}[linenos, frame=lines]{cypher}
MATCH p=shortestPath(
  (a:Actor {id: 1})-[*]-(b:Actor {id: 10})
  )
RETURN p;
  \end{minted}
  \citeurl{neo4j.com/docs/stable/cypher-query-lang.html}
\end{frame}

